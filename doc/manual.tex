\documentclass[12pt, a4paper, twocolumn]{report}
\usepackage[utf8]{inputenc}
\usepackage[russian]{babel}
\usepackage{hyperref}
\usepackage[]{graphicx}

%\makeindex

\newcommand{\PROGNAME}{stress-strain-assembly}
\newcommand{\IMPORTANT}{{\bf ВНИМАНИЕ:~}}

\newcommand{\CTL}[1]{<<{\bf #1}>>}

\newcommand{\CMD}[1]{<<{\tt #1}>>}

\newcommand{\FILENAME}[1]{{\tt #1}}

\newcommand{\PARAM}[1]{\item {\bf #1} }

\newcommand{\PARAMSECTION}[1]{\vbox{}{\bf Раздел <<#1>>}}

\input{./title.tex}
\author{Накин~А.~В.\\\href{mailto:andrey.nakin@gmail.com}{{\tt andrey.nakin@gmail.com}}}

\begin{document}

\maketitle

\tableofcontents

\chapter{Общие сведения}

Установка представляет собой программно-аппаратный комплекс для регистрации деформации и напряжения упругих материалов (далее~--- Образцов) в зависимости от приложенной нагрузки и температуры.

Установка производит съём показаний измерительных приборов, вычисление значений деформации и напряжения со всеми сопутствующими погрешностями и запись результатов в файлы данных.

Управление температурой Образца производится внешними устройствами. Установка сама никак не влияет на температуру Образца, а только регистрирует её.

Одновременно Установка способна работать только с одним единственным Образцом.

\section{Состав Установки}

\subsection{Состав аппаратной части}

Аппаратная часть Установки состоит из персональной ЭВМ (далее~--- ПЭВМ) и следующих приборов:

\begin{itemize}

\item Датчики угла поворота AC36/1212AR.41SBB компании \href{http://www.skbis.ru/}{СКБ ИС} (2~шт.). Предназначены для определения углов поворота.

\item Декодеры угла поворота ЛИР-916 компании \href{http://www.skbis.ru/}{СКБ ИС} (2~шт.). Подключаются к датчикам угла поворота и преобразуют их показания в удобную для использования форму. Управляются через шину RS-485 по протоколу Modbus-RTU.

\item Измеритель-регулятор одноканальный ТРМ-201 компании \href{http://www.owen.ru/}{ОВЕН} (1~шт.). Предназначен для определения температуры Образца посредством термопары. Управляется через шину RS-485 по протоколу Modbus-RTU.

\item Преобразователь интерфейсов USB/RS-485 АС-4 компании ОВЕН (1~шт.). Предназначен для сопряжения устройств, работающих на шине RS-485, с  ПЭВМ посредством USB интерфейса.

\end{itemize}

\subsection{Состав программной части}
\label{sec_software}

Программная часть Установка представляет собой исполнимый файл {\tt stress-strain-assembly.exe}, предназначенный для запуска в операционной системе Windows XP или выше.

\section{Принцип работы}

Образец закрепляется без люфта между валами №~1 и №~2. Каждый вал связан с собственным датчиком угла поворота. Вал №~1 приводится в движение электродвигателем, к валу №~2 присоединена статическая нагрузка с известным моментом. В непосредственной близости от Образца располагается рабочий спай термопары. Образец может нагреваться или охлаждаться.

Установка периодически с заданной частотой измеряет углы поворота валов и температуру Образца, вычисляет деформацию и напряжение и записывает результаты в файлы, а также выводит на экран ПЭВМ для оперативного контроля.

Оператор вручную управляет вращением вала №~1 и температурой Образца.

\subsection{Методы измерения}
\label{sec_measurement_method}

Установка может работать в двух основых режимах:

\subsubsection{Динамический момент}

В этом режиме к валу №~2 присоединяется статическая нагрузка --- рычаг с весом. Нагрузка имеет известный максимальный момент. Вал может свободно вращаться, и в зависимости от угла поворота меняется степень влияния нагрузки на Образец.

\subsubsection{Постоянный момент}
\label{sec_const_momentum}

В этом режиме вал №~2 фиксируется.

\subsection{Вычисляемые величины}

По значениям углов поворота валов Установка вычисляет деформацию и напряжение в Образце.

Деформация $\gamma$ вычисляется по формуле:

\begin{equation}
\label{eq_gamma}
\gamma = \frac{r}{l} \left( \phi_1 - \phi_2 \right),
\end{equation}

\noindent где $r$~--- радиус Образца, $l$~--- длина Образца, $\phi_1$ и $\phi_2$~--- углы поворота валов в радианах. Деформация выражается в процентах.

Напряжение в Образце $\tau$ вычисляется по формуле:

\begin{equation}
\label{eq_tau}
\tau = \frac{1.5 M_0 \sin \phi_2}{\pi r^3},
\end{equation}

\noindent где $M_0$~--- максимальное значение момента, приложенному к 
валу №~2. Напряжение выражается в Па.

В режиме \hyperref[sec_const_momentum]{постоянного момента}, когда вал №~2 зафиксирован, напряжение вычисляется по формуле:

\begin{equation}
\label{eq_tau}
\tau = \frac{1.5 M_0}{\pi r^3}.
\end{equation}

Величины $r$, $l$ и $M_0$ считаются постоянными на протяжении всего цикла измерений и вводятся Оператором перед началом эксперимента.

Величины $\gamma$ и $tau$ могут принимать отрицательные значения согласно формулам выше. Знак минуса не имеет физического смысла, но несёт информацию о <<переходе через ноль>> в ходе эксперимента, например при изменении направления вращения вала №~1. Поэтому Установка регистрирует $\gamma$ и $\tau$ <<как есть>>.

\subsection{Методы регистрации}

\label{sec_registration_types}

Частота регистрации (записи в файл) измерений вводится оператором перед началом измерений и может задаваться одним из следующих способов:

\begin{itemize}
\item {\bf Временная зависимость}~--- показания регистрируются с фиксированным временным интервалом, например один раз в секунду.
\item {\bf Температурная зависимость}~--- показания регистрируются с фиксированным температурным шагом, например $1$~Кельвин.
\item \label{sec_reg_type_manual} {\bf Вручную}~--- показания регистрируются по команде оператора.
\end{itemize}

\chapter{Подготовка к работе}

\section{Подготовка аппаратной части}

\subsection{Подготовка Образца и термопары}

Установите Образец в захваты, по возможности исключив люфт. Расположите рабочий спай термопары как можно ближе к Образцу. Проверьте подключение термопары к ТРМ-201.

\subsection{Подготовка АС-4}

Устройство должно быть подключено к ПЭВМ посредством интерфейса USB. На ПЭВМ должен быть установлен драйвер устройства. Если драйвер корректно установлен, в системе должен появиться новый COM-порт. Драйвер COM-порта можно скачать с \href{http://www.owen.ru/catalog/avtomaticheskij_preobrazovatel_interfejsov_usb_rs_485_owen_as4/opisanie}{сайта производителя}.

\IMPORTANT Устройство очень чувствительно к качеству соединения с ПЭВМ, поэтому рекомендуется подключать его непосредственно к USB разъёму ПЭВМ без использования промежуточных USB-разветвителей. Также необходимо использовать качественные USB-кабели с экранированием. В противном случае возможна потеря связи АС4 с ПЭВМ.

\subsection{Подготовка ТРМ-201}

\begin{enumerate}

\item Перед использованием устройство должно быть настроено на использование прокотола Modbus-RTU со следующими сетевыми установками:

\begin{itemize}
\item Длина адреса: 8~бит;
\item Скорость: 9600, 19200, 28800, 38400, 57600 или 76800 ~бод;
\item Чётность: none;
\item Стоповых бит: 1;
\end{itemize}

Настройка производится при помощи программы <<Конфигуратор ТРМ>>, которая идёт в комплекте с устройством. См. документацию на \href{http://www.owen.ru/catalog/izmeritel_regulyator_odnokanal_nij_s_rs_485_oven_trm201/opisanie}{сайте производителя}.

Сетевые настройки, будучи установленными, сохраняются в энергонезависимой памяти устройства и не теряются при отключении питания.

\item Выберите тип используемой термопары с передней панели устройства (см. документацию к ТРМ-201).

Тип термопары, будучи установленным, сохраняется в энергонезависимой памяти устройства и не теряется при отключении питания.

\item Проверьте подключение устройства к устройству АС-4 посредством шины RS-485.

\item Подайте питание на прибор. Должны загореться индикатор <<Питание>> на передней панели прибора и высветиться текущая температура рабочего спая в градусах Цельсия.

\end{enumerate}

\subsection{Подготовка ЛИР-916}

\begin{enumerate}

\item Перед использованием установите нужную скорость обмена по шине RS-485. Рекомендуется, чтобы все устройства на шине имели одинаковую скорость обмена и прочие сетевые настройки (количество битов данных, количество стоповых битов, чётность).

См. документацию к устройству на \href{http://www.skbis.ru/index.php?p=3&c=8&d=56}{сайте производителя}.

Сетевые настройки, будучи установленными, сохраняются в энергонезависимой памяти устройства и не теряются при отключении питания.

\item Проверьте подключение устройства к устройству АС-4 посредством шины RS-485.

\item Подайте питание на прибор.

\end{enumerate}

\subsection{Подготовка датчиков угла поворота}

Проверьте подключение датчика к ЛИР-916.

\section{Подготовка программной части}

Запустите программу Установки.

Убедитесь в работоспособности Программы и всех аппаратных устройств. Для этого убедитесь, что Программа определяет и выводит на экран ПЭВМ температуру Образца и показания с датчиков угла поворота.

Попробуйте немного провернуть валы: показания обоих датчиков должны меняться в ожидаемом диапазоне значений. Если показания датчиков меняются, но значения углов неверные, проверьте правильность коэффициентов пересчёта угла поворота.

\subsection{Параметры Образца}

Откройте закладку \CTL{Образец}. Данная вкладка содержит параметры измеряемого образца.

\subsubsection{Геометрические параметры}
\label{sec_geom_params}

Для вычисления деформации и напряжения необходимо ввести радиус и длину Образца, а также максимальное значение приложенного момента $M_0$.

Для каждой величины можно ввести соответствующую погрешность, которая будет учтена в расчётах. Если погрешность неизвестна или может быть проигнорирована, оставьте соответствующие поля пустыми или введите там нулевые значения.

\subsubsection{Файлы}

В поле \CTL{Имя файла результатов} вводится путь и имя файла, в который будут записаны результаты измерений для последующего использования. Как правило, это поле обязательно для заполнения. Если поле пустое, то в процессе измерений их результаты не будут фиксироваться в файле.

Имя файла может включать в себя строчку \CMD{\%AUTODATE\%} (без кавычек). Тогда путь к реальному файлу будет в этом месте включать три
 подкаталога, соответствующих текущим году, месяцу и дню месяца. Например, если в поле введено \CMD{C:\textbackslash{}res\textbackslash{}\%AUTODATE\%\textbackslash{}res.txt}, и измерения проводятся 5 января 2012 года, то реальный путь к файлу будет \CMD{C:\textbackslash{}res\textbackslash{}2012\textbackslash{}01\textbackslash{}05\textbackslash{}res.txt}.

В поле \CTL{Формат файлов} выбирается нужный формат всех создаваемых файлов. В настоящее время поддерживается два формата:

\begin{itemize}
\item \CTL{TXT} --- текстовый формат, в котором каждое измерение записывается в отдельной строке, внутри строки значения разделяются символом табуляции. Формат поддерживается большинством научных приложений.
\item \CTL{CSV} --- текстовый формат, в котором каждое измерение записывается в отдельной строке, внутри строки значения разделяются запятой. Формат поддерживается рядом офисных приложений, например Miscosoft Excel.
\end{itemize}

Флаг \CTL{Переписать файлы} указывает, нужно ли всякий раз в начале измерений переписывать файлы заново. Если флаг сброшен, все новые показания будут дописываться в конец файлов.

В поле \CTL{Комментарий}\label{sec_dut_comment} вводится произвольный фрагмент текста, который будет записан в начале файла результатов. Поле может быть пустым, но рекомендуется вводить в него краткое описание Образца и условия проведения измерений, это в будущем облегчит анализ результатов.

\subsection{Параметры измерения}

Откройте вкладку \CTL{Параметры измерения}. Содержимое данной вкладки определяет условия проведения измерений.

\subsubsection{Метод регистрации}
\label{sec_reg_method}

В данном разделе выбирается способ и частота регистрации измерений (см. описание принципа работы на стр.~\pageref{sec_registration_types}).

Если требуется регистрация с фиксированным временным интервалом, то выберите флаг \CTL{Временная зависимость}, после чего в поле \CTL{Временной шаг} введите значение интервала.

Если требуется регистрация с фиксированным температурным интервалом, то выберите флаг \CTL{Температурная зависимость}, после чего в поле \CTL{Температурный шаг} введите значение интервала.

Если требуется нерегулярная регистрация, то выберите флаг \CTL{Вручную}.

\subsubsection{Метод измерения}

Выберите метод измерения напряжения в соответствии с условиями эксперимента. (см. раздел <<Методы измерения>> на стр.~\pageref{sec_measurement_method})

\chapter{Измерения}

После подготовки к работе аппаратной и программной частей Установки можно приступить к измерениям.

\section{Проведение измерений}

Выберите вкладку \CTL{Измерение}. Ещё раз убедитесь в том, что все приборы работают, величины температуры и углов поворота  измеряются и находятся в ожидаемом диапазоне значений.

Перед началом измерения, когда Образец находится в ненапряжённом состоянии, необходимо обнулить показания датчиков угла поворота. Это необходимо для корректного вычисления деформации и напряжения. Для этого нажмите кнопку \CTL{Обнулить углы}.

После этого нажмите кнопку \CTL{Начать запись}. Установка начнёт регистрацию измеряемых и вычисляемых величин.

Оператор управляет скоростью и направлением поворота вала №~1 и температурой образца.

Если выбран режим ручной регистрации (стр.~\pageref{sec_reg_type_manual}), то оператор должен самостоятльно нажимать кнопку \CTL{Снять точку}\label{sec_manual} всякий раз, когда требуется зафиксировать измерение.

Кнопка \CTL{Снять точку} доступна и в других режимах, при регистрации временной и температурной зависимостей. То есть даже при автоматической регистрации оператор может вручную зарегистрировать нужное измерение. Например, если выбранный временной интервал довольно велик, а оператор наблюдает <<интересное поведение>> Образца, то он может вручную зафиксировать текущее измерение, даже если временной интервал ещё не истёк.

По окончании измерений нажмите кнопку \CTL{Остановить запись}, после чего в течении нескольких (до 10) секунд программа зафиксирует все результаты в файлах. Файлы далее доступны для анализа.

После этого можно произвести новую серию измерений, или завершить работу Установки.

\section{Индикация измерений}

В процессе работы Установка производит снятие показаний приборов, обработку и отображение результатов на экране Программы. Частота, с которой производится опрос приборов, зависит от параметров измерения (см. раздел <<Метод регистрации>> на стр.~\pageref{sec_reg_method}), а также от скорости изменения температуры. Чем быстрее изменяется температура Образца, и чем меньше заданный температурный или временной интервал, чем чаще будут опрашиваться измерительные приборы.

Индикация измерений производится на вкладке \CTL{Измерение}.

\subsection{Диаграммы}

Четыре диаграммы показывают состояние основных регистрируемых величин:

\subsubsection{Зависимость T(t)}

На диаграмме изображается зависимость температуры Образца от времени, по горизонтальной оси отложено количество измерений (не реальных секунд). Также изображается линейная аппроксимация зависимости $T(t)$ в виде фиолетовой линии.

\subsubsection{Зависимость $\tau(\gamma)$}

На диаграмме изображается зависимость напряжения в Образце от деформации, измерения изображаются в виде зелёных точек, связанных друг с другом. Каждая залёная точка соответствует одному измерению, записанному в файл результатов. Если эксперимент достаточно продолжительный и точек становится слишком много, они автоматически прореживаются, разумеется только на графике.

Если регистрация производится с большой скоросьтю (быстрее, чем один раз в секунду), часть результатов будет не видна на диаграмме. В файл данных результаты попадают в любом случае.

\subsubsection{Зависимость $\gamma(T)$}

На диаграмме изображается зависимость деформации от температуры.

\subsubsection{Зависимость $\tau(T)$}

На диаграмме изображается зависимость напряжения от температуры.

\subsection{Текстовые поля}

Также индикация измерений производится в текстовых полях, вместе озаглавленных как \CTL{Результаты измерения}. Здесь выводятся следующие текущие значения:

\begin{itemize}
\item $\phi_1$ (в градусах, не в радианах!);
\item $\phi_2$ (также в градусах);
\item деформация $\gamma$ в процентах;
\item напряжение $\tau$ в МПа;
\item температура в кельвинах.
\end{itemize}

Все величины сопровождаются инструментальной погрешностью.

\section{Файл результатов}

В файле результатов в начале идёт строка с комментарием (стр.~\pageref{sec_dut_comment}), далее идут следующие поля:

\begin{enumerate}
\item \CMD{Date/Time} --- местные дата и время, в которое было произведено измерение, в формате \mbox{\CMD{ГГГГ-ДД-ММ чч:мм:сс}}.
\item \CMD{T} --- температура Образца в К.
\item \CMD{+/-} --- погрешность определения температуры в К. Здесь и далее под погрешностью подразумевается абсолютная инструментальная погрешность.
\item \CMD{dT/dt} --- скорость изменения температуры в К/мин.
\item \CMD{phi1} --- $\phi_1$ в радианах.
\item \CMD{+/-} --- погрешность $\phi_1$ в радианах.
\item \CMD{phi2} --- $\phi_2$ в радианах.
\item \CMD{+/-} --- погрешность $\phi_2$ в радианах.
\item \CMD{gamma} --- деформация $\gamma$ в процентах.
\item \CMD{+/-} --- погрешность $\gamma$ в процентах.
\item \CMD{tau} --- напряжение $\tau$ в Па.
\item \CMD{+/-} --- погрешность $\tau$ в Па.
\end{enumerate}

\chapter{Завершение работы}

Для завершения работы Установки нажмите кнопку \CTL{Выход} в панели Программы. Программа произведёт сброс всех устройств в исходное состояние и закончит работу. Далее можно обесточить приборы и демонтировать Образец.

\chapter{Настройка Программы}

При первом запуске Программы, а также при всяком изменении аппаратной части, требуется произвести настройку или перенастройку программной части, чтобы обеспечить связь с аппаратной частью и, возможно, её калибровку.

\section{Файл конфигурации}

Все свои настройки программа сохраняет в обычном файле с именем \FILENAME{\PROGNAME{}.ini}, который размещается в текущем каталоге. Файл имеет текстовый формат, и при необходимости его можно редактировать в произвольном текстовом редакторе. Если файл конфигурации отсутствует при запуске Программы, он будет автоматически создан.

Обратите внимание: файл конфигурации всегда располагается в {\it текущем} каталоге, а не в каталоге, где расположена сама программа. Эти каталоги могут совпадать, но вообще говоря они могут быть разными. Это позволяет запускать одну и ту же программу, но с разными настройками для разных экспериментов.

Рассмотрим пример. Пусть имеется две измерительные установки, различающиеся, например, сборкой, в которой установлен Образец. Таким образом имеем одинаковый набор мультиметров и источников питания, но в разных экспериментах у нас разные термопары, имеющие разную калибровку. Чтобы каждый раз при переключении не менять настройки, мы делаем следующее:

\begin{enumerate}
\item Создаём два разных каталога, например \FILENAME{C:\textbackslash{}config\textbackslash{}exp1} и \FILENAME{C:\textbackslash{}config\textbackslash{}exp2}.

\item Сама программа путь будет расположена в каталоге \FILENAME{C:\textbackslash{}prog\textbackslash{}assembly007}.

\item Для работы с первой сборкой и переходим в каталог \FILENAME{C:\textbackslash{}config\textbackslash{}exp1} и запускаем программу оттуда. В этом же каталоге будет создан файл конфигурации \FILENAME{\PROGNAME{}.ini}.

\item Для работы со второй сборкой и переходим в каталог \FILENAME{C:\textbackslash{}config\textbackslash{}exp2} и производим аналогичные действия.

\end{enumerate}

Если мы работаем в операционной системе семейства Windows, то удобно будет создать ярлыки для каждой из сборок. В свойствах ярлыка укажите разные рабочие каталоги, тогда при выборе ярлыка будет устанавливаться соответствующий текущий каталог, в котором программа будет искать конфигурационный файл.

\section{Настройка основных параметров приборов}

Откройте вкладку \CTL{Параметры приборов}. Здесь оператор вводит параметры (адреса и пр.) устройств, которые используются для определения сопротивления Образца.

\subsection{Преобразователь интерфейса АС4}

Здесь оператор вводит имя последовательного порта, по которому осуществляется связь с АС4 и далее со всеми устройствами на шине RS-485. Это тот самый <<виртуальный>> порт, который появляется после подключения АС4 к ПЭВМ.

После ввода порта рекомендуется проверить, что порт доступен для Программы, для этого нажмите кнопку \CTL{Опрос}. Программа произведёт попытку открытия порта.

\subsection{Измеритель температуры ТРМ-201}

Здесь оператор вводит адрес устройства на шине RS-485 и скорость обмена. Предполагается, что прочие сетвые настройки таковы:

\begin{itemize}
\item битов данных: 8
\item стоповых битов данных: 1
\item контроль чётности: нет
\end{itemize}

Далее рекомендуется проверить, что устройство работает и доступно для Программы, для этого нажмите кнопку \CTL{Опрос}. Программа произведёт попытку прочитать текущую температуру.

\subsection{Декодер угла поворота ЛИР-916}

Здесь оператор вводит адрес устройства на шине RS-485 и скорость обмена.

Далее рекомендуется проверить, что устройство работает и доступно для Программы, для этого нажмите кнопку \CTL{Опрос}. Программа произведёт попытку прочитать текущий угол поворота.

\section{Калибровка приборов}

\subsection{Калибровка датчиков угла поворота}

Перед началом измерений необходимо вычислить коэффициент пересчёта угла поворота для каждого датчика поотдельности. Дело в том, что скорость вращения вала на Образце и вала датчика могут отличаться. Для этого выполните следующие действия:

\begin{enumerate}
\item откройте вкладку \CTL{Калибровка приборов} и нажмите кнопку \CTL{Определить коэффициент} для одного из датчиков.

\item Зафиксируйте вал (№~1 или №~2, далее просто <<вал>>) на Образце, чтобы он не двигался и нажмите кнопку \CTL{Дальше}. Программа произведёт считывание текущего показания датчика, назовём его $\alpha$.

\item Прокрутите вал на определённое число оборотов $n$, лучше целое и вновь зафиксируйте, после чего нажмите кнопку \CTL{Дальше}. Программа произведёт считывание текущего показания датчика, назовём его $\beta$. 

Направление вращения --- в сторону увеличения угла поворота.

\item Введите в поле число произведённых оборотов и нажмите кнопку \CTL{Сохранить}. Программа вычислит коэффициент пересчёта по формуле:

\begin{equation}
c = \frac{2 \pi n}{ \beta - \alpha }.
\end{equation}

Коэффициент угла поворота может быть отрицательным, если <<положительное>> направление вращения вала приводит к <<отрицательному>> направлению вращения датчика угла поворота.

\end{enumerate}

Данную операцию желательно выполнить для обоих датчиков поотдельности. Если передаточное число одинаково для обоих валов, то просто скопируйте значение коэффициента пересчёта в поле для другого датчика.

При всяком изменении коэффициента пересчёта перезапустите программу, чтобы изменения возымели эффект.

\subsection{Калибровка термопары}

Для более точного определения температуры можно указать корректировку показаний термопары --- арифметическое выражение, меняющее измеренную температуру должным образом.

Выражение включает в себя переменную, которая содержит в себе температуру, полученную непосредственно от термопары, одно или несколько чисел и арифметические операции, выражаемые символами \CMD{+}, \CMD{-}, \CMD{*} и \CMD{/}. Порядок операций~--- как принято в арифметике, для изменения порядка используются круглые скобки. Переменная обозначается произвольной латинской буквой. Между операндами допускается произвольное количество пробелов для улучшения читабельности.

{\bf Пример. } Пусть показания термопары дают температуру, увеличенную на $0.5$~К по сравнению с реальной, корректировка должна уменьшать исходное значение на $0.5$. Тогда выражение для корректировки будет \mbox{\CMD{x - 0.5}} (выражение вводится в поле без кавычек).

\section{Завершение настройки}

По окончании настройки Программа должна быть перезапущена, чтобы именения вошли в силу.

\chapter{Устранение неиправностей}

\section{Общие рекомендации}

Если при запуске Программы результаты измерений не отображаются, то рекомендуется выполнить следующие действия:

\begin{enumerate}
\item Убедитесь в правильности всех подключений.
\item Убедитесь, что Программа установила связь со всеми устройствами. Для этого зайдите на вкладку \CTL{Параметры установки} и выполните опрос всех используемых устройств по очереди.
\end{enumerate}

В случае обнаружения неверных настроек (например, выбран неверный метод измерения или введён неправильный адрес устройства), выполните их коррекцию и перезапустите Программу.

Если все вышеуказанные действия не помогли, необходимо ознакомиться с содержимым файла протокола для детального выяснения неисправности.

\section{Файл протокола}

Программа записывает информацию о всех обнаруженных неисправностях аппаратной части в файл протокола с именем \FILENAME{\PROGNAME{}.log}, который размещается в текущем каталоге. Файл всегда пополняется, то есть при обнаружении ошибки новые записи добавляются в конец файла.

Файл имеет простой текстовый формат, в котором каждая запись имеет следующий вид:

\CMD{Время Важность Модуль Описание}

Здесь <Время>~--- точные дата и время обнаружения неисправности. <Важность>~--- важность ситуации, которая может принимать следующие значения:

\begin{itemize}
\item \CTL{critical} --- критическая ошибка;
\item \CTL{error} --- важная ошибка;
\item \CTL{warning} -- предупреждение о возможной ошибке;
\item \CTL{info} --- информационное сообщение, не сигнализирующее об ошибке;
\item \CTL{debug} --- отладочное сообщение, предназначенное для отладки программы.
\end{itemize}

<Модуль>~--- имя модуля Программы, в котором обнаружена ошибка. <Описание>~--- произвольное текстовое описание, детально раскрывающее суть и местоположение ошибки. Описание может распологаться на нескольких строках файла протокола.

Если файл протокола отсутствует при работающей программе, значит в процессе работы ещё не возникало ни одной ошибки.

Файл протокола можно безопасно удалять, он будет вновь создан при первой же возникшей ошибке.

Поскольку файл протокола постоянно пополняется, оператору следует время от времени удалять его во избежание переполнения диска.

\end{document}
