\documentclass[12pt, a4paper]{article}
\usepackage[utf8]{inputenc}
\usepackage[russian]{babel}
\usepackage{hyperref}
\usepackage[]{graphicx}

\newcommand{\CTRL}[1]{<<{\bf #1}>>}

\begin{document}

\section*{Изменения в версии 1.0.1}

\subsection*{Коэффициенты угла поворота}

Коэффициент пересчёта угла поворота теперь задаётся для каждого датчика угла отдельно, для этого на вкладке \CTRL{Калибровка приборов} появилась новая секция.

Ранее коэффициент задавался только для датчика №~2 и принимался равным для обоих датчиков.

Перед началом измерений необходимо задать коэффициенты угла поворота для датчика №~1, для чего рекомендуется на вкладке \CTRL{Калибровка приборов} в секции \CTRL{Датчик угла поворота №~1} нажать кнопку \CTRL{Определить коэффициент}.

Если передаточное число одинаково для обоих датчиков, то можно просто скопировать коэффициент для датчика №~2 в соответствующее поле датчика №~1.

После определения коэффициентов перезапустите программу.

\subsection*{Исправленные ошибки}

\begin{itemize}
\item Исправлены перепутанные названия деформации и напряжения.
\item Исправлена ошибка чтения угла поворота из ЛИР-916.
\item Исправлена ошибка регистрации с температурным шагом.
\item Из формул расчёта $\tau$ и $\gamma$ убрано взятие по модулю, вследствие чего величины $\tau$ и $\gamma$ теперь могут принимать отрицательные значения, которые не имеют физического смысла, но сигнализируют о <<переходе через ноль>> в ходе эксперимента.
\end{itemize}

\end{document}
