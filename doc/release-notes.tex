\documentclass[12pt, a4paper]{article}
\usepackage[utf8]{inputenc}
\usepackage[russian]{babel}
\usepackage{hyperref}
\usepackage[]{graphicx}

\newcommand{\CTRL}[1]{<<{\bf #1}>>}
\newcommand{\VERSION}[1]{\section*{Изменения в версии #1}}
\newcommand{\ITEM}[1]{\subsection*{#1}}
\newcommand{\FIXEDERRORS}{\subsection*{Исправленные ошибки}}

\begin{document}

\VERSION{1.1.1}

\ITEM{Чтение из ЛИР-916}

Анализ логов показал, что иногда ЛИР-916 не возвращает информацию об угле поворота. Причины этого пока неясны. Теперь программа производит до 3 попыток чтения из ЛИР-916, в случае неудачи генерируется ошибка, и текущие измерения прекращаются.

\VERSION{1.1.0}

\ITEM{Оповещение о событиях}

Добавлено звуковое оповещение о событиях в процессе измерения, см. соответствующий раздел документации.

\ITEM{Предупреждение о том, что файл результатов существует}

Перед началом измерений Программа проверяет наличие на диске файла с результатами. Если файл существует, Программа выдаёт предупреждающее сообщение и предлагает подтвердить начало измерений.

\VERSION{1.0.2}

\ITEM{Коэффициент угла поворота}

Исправлена процедура вычисления коэффициента пересчёта угла поворота. Теперь коэффициент может принимать отрицательные значения.

\FIXEDERRORS

\begin{itemize}
\item Исправлена ошибка в процедуре вычисления коэффициента пересчёта угла поворота.
\end{itemize}

\VERSION{1.0.1}

\ITEM{Коэффициенты угла поворота}

Коэффициент пересчёта угла поворота теперь задаётся для каждого датчика угла отдельно, для этого на вкладке \CTRL{Калибровка приборов} появилась новая секция.

Ранее коэффициент задавался только для датчика №~2 и принимался равным для обоих датчиков.

Перед началом измерений необходимо задать коэффициенты угла поворота для датчика №~1, для чего рекомендуется на вкладке \CTRL{Калибровка приборов} в секции \CTRL{Датчик угла поворота №~1} нажать кнопку \CTRL{Определить коэффициент}.

Если передаточное число одинаково для обоих датчиков, то можно просто скопировать коэффициент для датчика №~2 в соответствующее поле датчика №~1.

После определения коэффициентов перезапустите программу.

\ITEM{Изменения в формулах}

Из формул расчёта $\tau$ и $\gamma$ убрано взятие по модулю, вследствие чего величины $\tau$ и $\gamma$ теперь могут принимать отрицательные значения, которые не имеют физического смысла, но сигнализируют о <<переходе через ноль>> в ходе эксперимента.

\FIXEDERRORS

\begin{itemize}
\item Исправлены перепутанные названия <<Деформация>> и <<Напряжение>>.
\item Исправлена ошибка чтения угла поворота из ЛИР-916.
\item Исправлена ошибка регистрации с температурным шагом.
\end{itemize}

\end{document}
